\documentclass[11pt,a4paper,oneside]{article}

\usepackage[utf8]{inputenc}
\usepackage[portuguese]{babel}
\usepackage[T1]{fontenc}
\usepackage{amsmath}
\usepackage{amsfonts}
\usepackage{amssymb}
\usepackage{multicol}

\usepackage{xcolor}
% Definindo novas cores
\definecolor{verde}{rgb}{0.25,0.5,0.35}
\definecolor{jpurple}{rgb}{0.5,0,0.35}
% Configurando layout para mostrar codigos Java
\usepackage{listings}
\lstset{
  language=Java,
  basicstyle=\ttfamily\small, 
  keywordstyle=\color{jpurple}\bfseries,
  stringstyle=\color{red},
  commentstyle=\color{verde},
  morecomment=[s][\color{blue}]{/**}{*/},
  extendedchars=true, 
  showspaces=false, 
  showstringspaces=false, 
  numbers=left,
  numberstyle=\tiny,
  breaklines=true, 
  backgroundcolor=\color{cyan!10}, 
  breakautoindent=true, 
  captionpos=b,
  xleftmargin=0pt,
  tabsize=4,
  escapeinside=||
}

\author{\\Universidade Federal de Goiás (UFG) - Câmpus Jataí\\Bacharelado em Ciência da Computação \\Lógica para Ciência da Computação \\Esdras Lins Bispo Jr.}

\title{\sc \huge Segundo Teste}

\begin{document}

\maketitle

{\bf ORIENTAÇÕES PARA A RESOLUÇÃO}

\begin{itemize}
	\item A avaliação é individual, sem consulta;
	\item A pontuação máxima desta avaliação é 10,0 (dez) pontos, sendo uma das 05 (cinco) componentes que formarão a média final da disciplina: dois testes, duas provas e exercícios;
	\item A média final será calculada pela média ponderada das cinco supraditas notas [em que o primeiro teste tem peso 20 (vinte), o segundo teste tem peso 10 (dez), a primeira prova tem peso 35 (trinta e cinco), a segunda prova tem peso 25 (vinte e cinco) e os exercícios têm peso 10 (dez)];
	\item O conteúdo exigido compreende os seguintes pontos apresentados no Plano de Ensino da disciplina: (4) Implicação Lógica e Argumento, (5) Demonstração e Dedução, e (6) Satisfazibilidade.
\end{itemize}

\begin{center}
	\fbox{\large Nome: \hspace{10cm}}
	\fbox{\large Assinatura: \hspace{9cm}}
\end{center}

\newpage

\begin{enumerate}

	\item (2,0 pt) [ESAF 2012] Conclua o argumento a seguir, marque a alternativa correta e {\bf justifique a sua resposta}. Se Marta é estudante, então Pedro não é professor. Se Pedro não é professor, então Murilo trabalha. Se Murilo trabalha, então hoje não é domingo. Ora, hoje é domingo. Logo,
	
		\begin{enumerate}
			\item Marta não é estudante e Murilo trabalha.
			\item Marta não é estudante e Murilo não trabalha.
			\item Marta é estudante ou Murilo trabalha.
			\item Marta é estudante e Pedro é professor.
			\item Murilo trabalha e Pedro é professor.
		\end{enumerate}
	
	\item (1,0 pt) Justifique cada passo na sequência de demonstração de $(q \rightarrow r) \wedge (s \vee \neg r) \wedge q\models s$:
	
		\begin{eqnarray*}
			(1) & q \rightarrow r & \\
			(2) & s \vee \neg r & \\
			(3) & q & \\
			(4) & r & \\
			(5) & s &  
		\end{eqnarray*}	
	
	\item (1,0 pt) Justifique cada passo na sequência de demonstração de $(p \rightarrow s) \wedge (p \rightarrow r) \models p \rightarrow (s \wedge r)$:
	
		\begin{eqnarray*}
			(1) & p \rightarrow s & \\
			(2) & p \rightarrow r & \\
			(3) & p & \\
			(4) & s & \\
			(5) & r & \\
			(6) & s \wedge r & 
		\end{eqnarray*}	
		
	\item (6,0 pt) Prove que os argumentos abaixo são válidos através do uso de regras de inferência:
	
		\begin{enumerate}
			\item (2,0 pt) $(p \rightarrow (q \vee r)) \wedge \neg q \wedge \neg r \models \neg p$
			\item (2,0 pt) $(p \rightarrow (q \rightarrow r)) \wedge (p \vee \neg s) \wedge q \models s \rightarrow r$			
			\item (2,0 pt) $(p \vee (q \rightarrow p)) \wedge q \models p$
		\end{enumerate}
	
\end{enumerate}

\newpage

{\sc \huge Material de Consulta}

\vspace{2cm}

REGRAS DE INFERÊNCIA

\begin{multicols}{2}
\begin{itemize}
	\item {\bf Silogismo Disjuntivo} (SD)
	\begin{eqnarray*}
		(1) & p \vee q& \\
		(2) & \neg p& \\
		 & \line(1,0){25} & \\
		(3) & q & \hspace{1cm} SD \mbox{ } (1),(2)
	\end{eqnarray*}
	\item {\bf De Morgan} ($DM_\wedge$)
			\begin{eqnarray*}
				(1) & \neg p \wedge \neg q & \\
				 & \line(1,0){25} & \\
				(2) & \neg (p \vee q) & \hspace{1cm} DM_\wedge \mbox{ } (1)
			\end{eqnarray*}
			
	\item {\bf De Morgan} ($DM_\vee$)
			\begin{eqnarray*}
				(1) & \neg p \vee \neg q & \\
				 & \line(1,0){25} & \\
				(2) &  \neg (p \wedge q) & \hspace{1cm} DM_\vee \mbox{ } (1)
			\end{eqnarray*}
\end{itemize}

\columnbreak

\begin{itemize}
	\item {\bf Modus Ponens} (MP)
	\begin{eqnarray*}
		(1) & p \rightarrow q & \\
		(2) & p & \\
		 & \line(1,0){25} & \\
		(3) & q & \hspace{1cm} MP \mbox{ } (1),(2)
	\end{eqnarray*}
	\item {\bf Modus Tollens} (MT)
			\begin{eqnarray*}
				(1) & p \rightarrow q & \\
				(2) & \neg q & \\
				 & \line(1,0){25} & \\
				(3) & \neg p & \hspace{1cm} MT \mbox{ } (1),(2)
			\end{eqnarray*}
			
	\item {\bf Contradição} ($\neg e$)
			\begin{eqnarray*}
				(1) & p & \\
				(2) & \neg p & \\
				 & \line(1,0){25} & \\
				(3) & \perp & \hspace{1cm} \neg e \mbox{ } (1),(2)
			\end{eqnarray*}
\end{itemize}
\end{multicols}


\begin{itemize}	
	\item {\bf Introdução da Conjunção} ($\wedge i$)
			\begin{eqnarray*}
				(1) & p & \\
				(2) & q & \\
				 & \line(1,0){25} & \\
				(3) & p \wedge q & \hspace{1cm} \wedge i \mbox{ } (1),(2)
			\end{eqnarray*}
\end{itemize}

\end{document}