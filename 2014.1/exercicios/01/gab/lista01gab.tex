\documentclass[12pt,a4paper,oneside]{article}

\usepackage[utf8]{inputenc}
\usepackage[portuguese]{babel}
\usepackage[T1]{fontenc}
\usepackage{amsmath}
\usepackage{amsfonts}
\usepackage{amssymb}
\usepackage{color}

\definecolor{verde}{rgb}{0,0.4,0}

\author{\\Universidade Federal de Goiás (UFG) \\Lógica para Ciência da Computação \\Esdras Lins Bispo Jr.}

\title{\sc \huge Lista de Exercícios 1}

\begin{document}

\maketitle

\begin{enumerate}

	\item Quais das frases a seguir são proposições?

		\begin{enumerate}
			\item A lua é feita de queijo verde. {\color{verde} É uma proposição}
			\item Ele é um homem alto. {\color{verde} É uma proposição}
			\item Dois é um número primo. {\color{verde} É uma proposição}
			\item O jogo terminará logo? {\color{red} Não é uma proposição}
			\item As taxas do ano que vem serão maiores. {\color{verde} É uma proposição}
			\item As taxas do ano que vem serão menores. {\color{verde} É uma proposição}
			\item $x - 4 = 0$ {\color{verde} É uma proposição}
		\end{enumerate}
	
	\item Simplifique as seguintes fórmulas, removendo os parênteses que não são obrigatórios:

		\begin{enumerate}
			\item $(p \vee q)$ {\color{verde} $\equiv p \vee q$}
			\item $((p \vee q) \vee (r \vee s))$ {\color{verde} $\equiv p \vee q \vee (r \vee s)$}
			\item $(p \rightarrow (q \rightarrow (p \wedge q)))$ 
			      {\color{verde} $\equiv p \rightarrow q \rightarrow p \wedge q$}
			\item $\neg (p \vee (q \wedge r))$ {\color{verde} $\equiv \neg (p \vee q \wedge r)$}
			\item $\neg (p \wedge (q \vee r))$ {\color{verde} $\equiv \neg (p \wedge (q \vee r))$}
			\item $((p \wedge (p \rightarrow q)) \rightarrow q)$ 
			      {\color{verde} $\equiv (p \wedge (p \rightarrow q)) \rightarrow q$}
		\end{enumerate}
		
	\item Adicione os parênteses às seguintes fórmulas para que fiquem de acordo com as regras de formação de fórmulas:
	
		\begin{enumerate}
			\item $\neg p \rightarrow q$ {\color{verde} $\equiv ((\neg p) \rightarrow q)$}
			\item $p \wedge \neg q \wedge r \wedge \neg s$ 
			      {\color{verde} $\equiv (((p \wedge (\neg q)) \wedge r) \wedge (\neg s))$}
			\item $p \rightarrow q \rightarrow r \rightarrow p \wedge q \wedge r$ 
			      {\color{verde} $\equiv (p \rightarrow (q \rightarrow (r \rightarrow ((p \wedge q) \wedge r))))$}
			\item $p \wedge \neg q \vee r \wedge s$ 
			      {\color{verde} $\equiv ((p \wedge (\neg q)) \vee (r \wedge s))$}
			\item $p \wedge \neg (p \rightarrow \neg q) \vee \neg q$ 
			      {\color{verde} $\equiv ((p \wedge (\neg (p \rightarrow (\neg q)))) \vee (\neg q))$}
		\end{enumerate}
	
	\item Dar o conjunto de subfórmulas das fórmulas a seguir (notar que os parênteses implícitos sao fundamentais para decidir quais são as subfórmulas):	
	
		\begin{enumerate}
			\item $\neg p \rightarrow p$ 
			     {\color{verde} \begin{eqnarray*}
										{\tt Subf}(\neg p \rightarrow p) & = \{ & \neg p \rightarrow p,\\
								      		& & \neg p,\\
								      		& & p \}
								 \end{eqnarray*}
				  }
			\item $p \wedge \neg q \wedge r \wedge \neg s$ 
				  {\color{verde} \begin{eqnarray*}
										{\tt Subf}(p \wedge \neg q \wedge r \wedge \neg s) & =  \{ & p \wedge \neg q \wedge r \wedge \neg s,\\
										& & p \wedge \neg q \wedge r, \\
										& & \neg s, \\
										& & p \wedge \neg q, \\
										& & r, \\
										& & s, \\
										& & p, \\
										& & \neg q, \\
										& & q \}
								\end{eqnarray*}
					}
			\item $p \rightarrow q \rightarrow r \rightarrow p \wedge q \wedge r$
				  {\color{verde} \begin{eqnarray*} 
				  {\tt Subf}(p \rightarrow q \rightarrow r \rightarrow p \wedge q \wedge r) & =  
				  \{ & p \rightarrow q \rightarrow r \rightarrow p \wedge q \wedge r,\\
				  		& & p,\\
				  		& & q \rightarrow r \rightarrow p \wedge q \wedge r,\\
				  		& & q,\\
				  		& & r \rightarrow p \wedge q \wedge r,\\
				  		& & r,\\
				  		& & p \wedge q \wedge r,\\
				  		& & p \wedge q\}
				  	\end{eqnarray*}
				  }
			\item $p \wedge \neg q \vee r \wedge s$
			{\color{verde} \begin{eqnarray*} 
				  {\tt Subf}(p \wedge \neg q \vee r \wedge s) & = \{ & p \wedge \neg q \vee r \wedge s,\\
				  		& & p \wedge \neg q,\\
				  		& & r \wedge s,\\
				  		& & p,\\
				  		& & \neg q,\\
				  		& & r,\\
				  		& & s,\\
				  		& & q\}
				  	\end{eqnarray*}
				  }
			\item $p \wedge \neg (p \rightarrow \neg q) \vee \neg q$
			{\color{verde} \begin{eqnarray*} 
				  {\tt Subf}(p \wedge \neg (p \rightarrow \neg q) \vee \neg q) & = \{ & p \wedge \neg (p \rightarrow \neg q) \vee \neg q,\\
				  		& & p \wedge \neg (p \rightarrow \neg q),\\
				  		& & \neg q,\\
				  		& & p,\\
				  		& & \neg (p \rightarrow \neg q),\\
				  		& & q,\\
				  		& & p \rightarrow \neg q\}
				  	\end{eqnarray*}
				  }
		\end{enumerate}
		
	\item Calcular a complexidade de cada fórmula do exercício anterior (notar que a posição exata dos parênteses {\it não influencia} a complexidade da fórmula).
	\begin{enumerate}
			\item {\color{verde} \begin{eqnarray*}
										|\neg p \rightarrow p| & = & |\neg p| + |p| + 1\\
								      			& = & |p| + 1 + 1 + 1\\
								      			& = &  1 + 3 \\
								      			& = & 4
								 \end{eqnarray*}
				  }
			\item {\color{verde} \begin{eqnarray*}
										|p \wedge \neg q \wedge r \wedge \neg s| & = & |p \wedge \neg q \wedge r| +  |\neg s| + 1\\
								      			& = & |p \wedge \neg q| + |r| + 1 +  |s| + 1 + 1\\
								      			& = & |p| + |\neg q| + 1 + 1 + 1 + 3\\
								      			& = & 1 + |q| + 1 + 6\\
								      			& = & 1 + 8\\
								      			& = & 9
								 \end{eqnarray*}
				  }
				  
			\item {\color{verde} \begin{eqnarray*}
										|p \rightarrow q \rightarrow r \rightarrow p \wedge q \wedge r| & = & |p| + |q \rightarrow r \rightarrow p \wedge q \wedge r| + 1\\
								      			& = & 1 + |q| + |r \rightarrow p \wedge q \wedge r| + 1 + 1\\
								      			& = & 1 + |r| + |p \wedge q \wedge r| + 1 + 3\\
								      			& = & 1 + |p \wedge q| + |r| + 5\\
								      			& = & 1 + |p| + |q| + 1 + 1 + 6\\
								      			& = & 1 + 1 + 9\\
								      			& = & 11\\
								 \end{eqnarray*}
				  }
			\item {\color{verde} \begin{eqnarray*}
										|p \wedge \neg q \vee r \wedge s| & = & |p \wedge \neg q| + |r \wedge s| + 1\\
								      			& = & |p| + |\neg q| + 1 + |r| + |s| + 1 + 1\\
								      			& = & 1 + |q| + 1 + 1 + 1 + 3\\
								      			& = & 1 + 7\\
								      			& = & 8
								 \end{eqnarray*}
				  }
			\item {\color{verde} \begin{eqnarray*}
										|p \wedge \neg (p \rightarrow \neg q) \vee \neg q| & = & |p \wedge \neg (p \rightarrow \neg q)| + |\neg q| + 1\\
								      			& = & |p| + |\neg (p \rightarrow \neg q)| + 1 + |q| + 1 + 1\\
								      			& = & 1 + |(p \rightarrow \neg q)| + 1 + 1 + 3\\
								      			& = & |p| + |\neg q| + 1 + 6\\
								      			& = & 1 + |q| + 1 + 7 \\
								      			& = & 1 + 9 \\
								      			& = & 10
								 \end{eqnarray*}
				  }
		\end{enumerate}
	
	\item Definir por indução sobre a estrutura de fórmulas a função {\it átomos}($A$), que retorna o conjunto de todos os átomos que ocorrem na fórmula $A$. Por exemplo, {\it átomos}($p \wedge \neg (p \rightarrow \neg q) \vee \neg q) = \{p,q\}$.
	
	{\color{verde}
	
		{\it átomos}($A$) é um conjunto definido como se segue:
			\begin{enumerate}
				\item {\bf Caso básico:} $p$ \\{\it átomos}($p$) = $\{ p \}$ para toda fórmula atômica $p \in \mathcal{P}$;
				\item {\bf Caso} $\neg A$ \\{\it átomos}($\neg A$) = {\it átomos}($A$);
				\item {\bf Caso} $|(A \circ B)|$ \\{\it átomos}($A \circ B$) = {\it átomos}($A$) $\cup$ {\it átomos}($B$), para $\circ \in \{ \wedge, \vee, \rightarrow \}$.
			\end{enumerate}
	
	}
	
%	\item Baseado nos símbolos proposicionais que estão no {\it slide} 15 da Aula 04 (24 de abril), expressar os seguintes fatos com fórmulas da lógica proposicional.
%	
%		\begin{enumerate}
%			\item Uma criança não é um jovem.
%			\item Uma criança não é jovem, nem adulto, nem idoso.
%			\item Se um adulto é trabalhador, então ele não está aposentado.
%			\item Para ser aposentado, a pessoa deve ser um adulto ou um idoso.
%			\item Para ser estudante, a pessoa deve ser ou um idoso aposentado, ou um adulto trabalhador ou um jovem ou uma criança.
%		\end{enumerate}
		
%	\item Sejam $p$, $q$ e $r$ as seguinte sentenças:
%
%		\begin{itemize}
%			\item[$p$:] Rosas são vermelhas.
%			\item[$q$:] Violetas são azuis.
%			\item[$r$:] Açúcar é doce.
%		\end{itemize}
%	
%	Traduza as seguintes sentenças compostas para notação simbólica:
%	
%		\begin{enumerate}
%			\item Rosas são vermelhas e violetas são azuis.
%			\item Rosas são vermelhas e; violetas são azuis ou açúcar é doce.
%			\item Sempre que violetas são azuis, as rosas são vermelhas e o açúcar é doce.
%			\item Rosas são vermelhas apenas se as violetas não forem azuis e se o açúcar for azedo.
%			\item Rosas são vermelhas e, se o açúcar for azedo, então as violetas não são azuis ou o açúcar é doce.
%		\end{enumerate}
	
\end{enumerate}

\end{document}