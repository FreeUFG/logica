\documentclass[12pt,a4paper,oneside]{article}

\usepackage[utf8]{inputenc}
\usepackage[portuguese]{babel}
\usepackage[T1]{fontenc}
\usepackage{amsmath}
\usepackage{amsfonts}
\usepackage{amssymb}

\author{\\Universidade Federal de Goiás (UFG) - Câmpus Jataí\\Bacharelado em Ciência da Computação \\Lógica para Ciência da Computação \\Esdras Lins Bispo Jr.}

\date{15 de abril de 2014}

\title{\sc \huge Primeiro Teste}

\begin{document}

\maketitle

{\bf ORIENTAÇÕES PARA A RESOLUÇÃO}

\begin{itemize}
	\item A avaliação é individual, sem consulta;
	\item A pontuação máxima desta avaliação é 10,0 (dez) pontos, sendo uma das 05 (cinco) componentes que formarão a média final da disciplina: dois testes, duas provas e exercícios;
	\item A média final será calculada pela média ponderada das cinco supraditas notas [em que o primeiro teste tem peso 20 (vinte), o segundo teste tem peso 10 (dez), a primeira prova tem peso 35 (trinta e cinco), a segunda prova tem peso 25 (vinte e cinco) e os exercícios têm peso 10 (dez)];
	\item O conteúdo exigido compreende os seguintes pontos apresentados no Plano de Ensino da disciplina: (1) Lógica Proposicional.
\end{itemize}

\begin{center}
	\fbox{\large Nome: \hspace{10cm}}
	\fbox{\large Assinatura: \hspace{9cm}}
\end{center}

\newpage

\begin{enumerate}

	\item (1,5 pt) Quais das frases a seguir são proposições? {\bf Justifique se não for.}

		\begin{enumerate}
			\item Vinte milhões é menor do que um.
			\item O Penapolense ganhou ontem?
			\item O professor de Lógica é muito bonito.
			\item Existem formas de vida em outros planetas do universo.
		\end{enumerate}
		
	\item (2,5 pt) Simplifique as seguintes fórmulas, removendo os parênteses que não são obrigatórios:

		\begin{enumerate}
			\item $((p \vee q) \vee (r \vee s))$ \ \ \ \ \ \ (0,5 pt)
			\item $(p \rightarrow (q \rightarrow (p \wedge q)))$ \ \ \ \ \ \ (1,0 pt)
			\item $((p \wedge (p \rightarrow q)) \rightarrow q)$ \ \ \ \ \ \ (1,0 pt)
		\end{enumerate}

	\item (4,5 pt) Dar o conjunto de subfórmulas das fórmulas a seguir:
	
		\begin{enumerate}
			\item $p \wedge (\neg p \vee \neg r)$  \ \ \ \ \ \ (1,0 pt)
			\item $p \vee q \rightarrow \neg r \wedge \neg s$  \ \ \ \ \ \ (1,5 pt)
			\item $p \wedge \neg (p \rightarrow \neg q) \vee \neg q$  \ \ \ \ \ \ (2,0 pt)
		\end{enumerate}	
		
	\item (1,5 pt) Calcular a complexidade de cada fórmula da questão anterior.
	
\end{enumerate}

\end{document}